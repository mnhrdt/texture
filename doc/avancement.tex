\documentclass{article}
\usepackage[utf8]{inputenc}         % to type accents
\usepackage[T1]{fontenc}            % for automatic french-style guillemets
\usepackage[british,french]{babel}  % french, english typographical conventions
\usepackage{amssymb}                % fancy symbols like \checkmark
\usepackage{color}                  % colorized text
\usepackage{url,hyperref}           % to make clickable hyperlinks
\usepackage[osf,sups]{Baskervaldx}  % fancy font
\usepackage[bigdelims,cmintegrals,vvarbb,baskervaldx]{newtxmath} % math font
\usepackage[cal=boondoxo]{mathalfa} % mathcal from STIX, unslanted a bit

% macros for math
\def\R{\mathbf{R}} % real numbers
\def\N{\mathbf{N}} % natural numbers

% macros for colorized comments
\newcommand{\mnhrdt}[1]{{\footnotesize\textcolor[rgb]{0.8,0.5,0.5}{eml: #1}}}

% smaller margins
\addtolength{\hoffset}{-3em}
\addtolength{\voffset}{-3em}
\addtolength{\textwidth}{6em}
\addtolength{\textheight}{6em}

\title{Avancement sur la texturation}
\author{Marie d'Autume}


\begin{document}
\maketitle

%%%%%%%%%%%%%%%%%
%%%%TO DO LIST%%%%
%%%%%%%%%%%%%%%%%
\section{Liste de travaux à faire}
\label{sec:todolist}

\begin{itemize}
\item[\checkmark] Création d'un mesh grossier à partir du lidar, coloré à l'aide d'une image : \verb"colorsingle.c"
\item[\checkmark] Création d'un mesh grossier à partir du lidar, coloré à partir de plusieurs images, choix de l'image la plus en face : \verb"colormultiple.c"
\item[$\sim$] z-buffer prenant en comptes les triangles, pas simplement les points du lidar (plus ou moins ok : il reste des probl\`emes que je ne comprends pas). Modifier le z-buffer pour qu'il accepte la projection de sommets voisins sur un même pixel.
\item création d'un mesh le plus propre possible pour le prototypage
\begin{itemize}
\item[\checkmark] régulariser le lidar (fait à l'aide du curvature microscope
sur ipol~\footnote{\url{http://www.ipol.im/pub/pre/212/}})
\item[\checkmark]  créer les faces triangulaires selon la hauteur des sommets et
	non toujours pareil
\item (19/07) regarder ``smoothing ... bilateral'' de Julie Digne sur ipol pour
régulariser les normales des faÀ§ades.
\end{itemize}
\item (19/07) recaler les images en utilisant les données de gabriele sur menthe \url{menthe:/home/facciolo/iarpa/all_pairs/outdir_from_19_32/ncc_transform.txt"}. %Exemple de mauvais recalage : Avec comme input le lidar r\'egularis\'e et les 9 premi\`eres images, colormultiple.c utilise la 5\`eme image pour l'arri\`ere de nombreux b\^atiments. Mais cette image est mal recal\'ee et on voit que les fa\c cades des b\^atiments sont \`a moiti\'e blanches (venant de zones satur\'ees de l'image correspondant aux toits des b\^atiments et non \`a leurs fa\c cades).
	\item (22/07) appliquer non local pansharpening aux images multispectrales pour obtenir des meshs en couleurs (JMM).

\item (19/07) STRUCTURES DE DONNÉS POUR LA FUSION DE TEXTURES.  
\begin{itemize}
\item[\checkmark] À chaque sommet, la texture de l'image visible la plus frontale
\end{itemize}
\mnhrdt{L'objectif de ce
travail est avoir un maillage avec une texture nouvelle qui n'existe
pas encore (donc, l'atlas avec toutes les images d'entrée est
juste une étape intermédiaire qui ne fait pas partie de la
sortie).  Il faut d'ores et déjà avoir un algorithme (même
naïf) qui crée un mesh de sortie avec une seule texture
obtenue par fusion des toutes les images d'entrée.  On devra
essayer plusieurs critères de fusion, et pouvoir les comparer
facilement sur des nouvelles donnés d'entrée.  Notamment
\begin{itemize}
	\item[\checkmark] À chaque sommet, la texture de l'image visible la plus frontale
	\item À chaque triangle, la texture de l'image visible la plus frontale
	\item À chaque triangle, la texture de l'image visible de plus haute
		résolution une fois projetée sur le triangle.
	\item À chaque triangle, la moyenne arithmétique de toutes les
		images visibles.
	\item À chaque triangle, la médiane de toutes les images visibles.
	\item À chaque triangle, la moyenne de toutes les images pondérée par
		une fonction monotone de l'angle de visibilité de chacune.
	\item Comme avant, mais avec médiane pondérée
	\item Autres critère de fusion de couleurs, ou fusion pondérée (e.g.,
		modes, k-medians, etc).
	\item À chaque triangle, la moyenne des gradients de toutes les
		images visibles.  Puis résoudre l'équation de Poisson sur la
		surface entière pour récuperer une texture.
	\item À chaque triangle, la médiane (ou d'autres fonctions robustes,
		pondérées ou non,  des gradients de toutes les images
		visibles).  Puis résoudre l'équation de Poisson.
	\item Comme avant, mais en combinant drift-fields et utilisant
		l'équation d'Osmose sur la surface.
	\item Tous les critères antérieurs, mais utilisant un changement de
		contraste~$I_i\mapsto\alpha_i I_i+\beta_i$ sur chaque image,
		où les champs de coefficients $\alpha$ et $\beta$ sont très
		lisses et utilisées pour harmoniser les couleurs le plus
		possible avant de la fusion.  (Ce critère est nouveau, par
		rapport aux articles antérieurs, il me parait)
\end{itemize}
Il me parait que ce point est le plus important.  Au moins, il faut implémenter deux ou trois critères de fusion de
textures et pouvoir comparer les ply texturisées de sortie.  Ceci est le plus
urgent.  Ensuite, on pourra améliorer la qualités des maillages, du recalage
des images, etc.}
\end{itemize}

 %%%%%%%%%%%%%%%%%
%%%%%%%REMARQUES%%%%%%%
 %%%%%%%%%%%%%%%%%
\section{Remarque}
Pourquoi dans la structure de données dis-tu à chaque triangle ? Je croyais que nous allions travailler sur les sommets. Faut-il que j'attribue une même couleur à chaque triangle plutôt ? 


 %%%%%%%%%%%%%%%%%
%%%%%%%AVANCEMENT%%%%%%%
 %%%%%%%%%%%%%%%%%
\section{Résumé des activités de la journée}
\paragraph{19/07/2017} création de ce pdf, essai de \verb"colormultiple.c" sur le lidar régularisé, découpage plus intelligent de face triangulaire. \c Ca marche, on devine des fenêtres ! travail en cours sur zbuffer.c (amlélioration du z-buffer de colorize\_with\_shadows.c)
\paragraph{20/07/2017}  travail toujours en cours sur zbuffer.c (j'ai passé la journée à faire du debuggage et à améliorer ma connaissance des pointeurs). zbuffer.c est plus ou moins fonctionnel. 
\paragraph{21/07/2017} nettoyage du main de colormultiple pour le séparer en sous ensembles. Début de travail sur le recalage.
\paragraph{24/07/2017} recalage en cours (j'espère avoir des résultats demain mais cela m'étonnerait qu'ils soient bons : les vecteurs de déplacement semblent trop élevés). étude de pansharpening pendant que la grande image ntf charge.
\paragraph{25/07/2017}  modification de colormultiple.c pour permettre des images couleurs. modification de get\_corners.c pour permettre le pansharpening (les crops ont maintenant des dimensions multiples de 4). Début de travail sur colorfancy.c pour à terme colorer les sommets et non les faces. 
\paragraph{26/07/2017} Comme je l'avais deviné, le recalage par simple projection des vecteurs de déplacement ne marche pas. Colorfancy marche et renvoie pour chaque sommet la couleur de l'image la plus frontale.
\paragraph{27-28/07/2017} toutes les images msi  semblent avoir un offset proche de (13, -0,375) par rapport au pan. Problème de décalage et de saturation des images couleurs avec croprgb.sh. (saturation : travailler avec des floats, décalage : utiliser l'interpolation avec gdal\_translate).  Gabriele a relancé un scripts avec les bons paramètres et j'ai du coup les bons vecteurs de déplacement dans l'espace des nuages de points : voir ce qu'ils donnent projetés. En vue du sur-échantillonage du mesh, modification projetée du zbuffer pour qu'il autorise plusieurs sommets voisins à se projeter sur le même pixel de l'image.


\clearpage
\begin{otherlanguage}{british}
\section{Data structures for surfaces, and for functions on surfaces}

\subsection{Continuous setting}

A surface is a smooth 2-dimensional sub-manifold of~$\R^3$.
There are three common ways to define a surface~$S$:
\begin{itemize}
	\item \emph{Implicit} representation by a function~$f:\R^3\to\R$,
		so that~$S=f^{-1}(\{0\})$.
	\item \emph{Parametric} representation by a function~$s:\R^2\to\R$,
		so that~$S=s(\R^2)$.

	\item By a set of \emph{coordinate charts} $(U_i,\varphi_i)$,
		where~$S=\bigcup U_i$, $\varphi_i:U_i\to\R^2$ and the transition
		functions~$\varphi_i\circ\varphi_j^{-1}:\R^2\to\R^2$ are
		smooth.  A set of coordinate charts is called
		an~\emph{atlas}.
\end{itemize}
The implicit and atlas representations allow for arbitrary topologies.  The
parametric representation is the simplest one, but it can only represent
surfaces homeomorphic to a square.  An even more restricted parametric
representation is the 2.5D representation, where the parametrization $s(x,y)$
is of the form~$s(x,y)=(x,y,z(x,y))$ where~$z:\R^2\to\R$.

When working with surfaces, we usually work with functions~$f:S\to\R$.
There are different ways to define a function~$f:S\to\R$.  One possibility is
to define a function $g:\R^3\to\R$, and then set~$f=g|_S$.  However, this
representation is non-unique, since many different~$g$ can give the same~$f$.
Another possibility, if~$S$ is defined by an atlas~$\{U_i,\varphi_i\}$, is to
give the functions~$f_i=f\circ\varphi_i^{-1}:\R^2\to\R$.
%If~$S\subset\R^3$ is a surface, a function~$f:S\to\R$ is said to be smooth
%when it is the restriction to~$S$ of a smooth function of~$\R^3$.  To check
%that a function is smooth (or other properties of the function), it depends
%on how the surface is defined (by parametrization or by atlas).
%\begin{itemize}
%	\item If~$S$ is defined by a parametrization~$s:\R^2\to\R^3$, we need
%		to check that the function~$f\circ s:\R^2\to\R$ is smooth.
%	\item If~$S$ is defined by an atlas~$\{U_i,\varphi_i\}$, we need to
%		check that the functions~$f\circ\varphi_i^{-1}:\R^2\to\R$ are
%		smooth.
%\end{itemixe}
Notice, however, that an arbitrary collection of functions~$f_i:\R^2\to\R$
does~\emph{not} in general define a function on~$S$ via this atlas.  A
compatibility criterion is required, whereby
\begin{equation}\label{eq:compat}
f_i\circ\varphi_i=f_j\circ\varphi_j\quad\textrm{when}\quad U_i\cap U_j\neq\emptyset
\end{equation}
		(the
local definitions the function must be compatible on overlapping charts).

A standard construction in differential geometry is the Laplace-Beltrami
operator~$\Delta_g$ on a surface~$S$, which is a generalization of the Laplacian
operator~$\Delta=\frac{\partial^2}{\partial x^2}+\frac{\partial^2}{\partial
y^2}$ in the euclidean plane.  It allows to state, for example, Poisson
equation on the surface
\[
	\Delta_g u = f
\]
that has a unique solution up to an additive constant.

In a coordinate chart, the Laplace-Beltrami operator has the
form~$\alpha\mathrm{div}\left(A\nabla u\right)$.  This is an anisotropic
diffusion operator where the matrix field~$A$ and scalar field~$\alpha$ are
formed by the coefficients of the first fundamental form of the surface
(derivatives of the parametrization).

Similarly, you can define gradients and divergences for functions defined on
the surface, and state Osmosis equation, and so on.


\subsection{Discretization of a surface}

The standard way to discretize a surface is to use a mesh of
triangles~\footnote{simple or low-resolution surfaces can also be represented
very efficiently by implicit functions, and this allows for very fast
computer graphics operations.  However, it does not scale well when you need a
lot of geometric detail, so we will ignore this technique here}.
A mesh of~$n$ vertices and~$m$ triangles is represented by two lists:
\begin{enumerate}
	\item A list of~$n$ points~$P=\{(x_i,y_i,z_i)\}_{i=1,\ldots,
		n}\subset\R^3$ giving the spatial positions of the vertices.
	\item A list of~$m$ triangles~$T=\{(a_j,b_j,c_j)\}_{j=1,\ldots,
		m}\subset\N^3$ giving the indices of the vertices of each
		triangle (as positions inside the list~$P$).
\end{enumerate}

There are also other, fancier data structures, for example~\emph{triangle
strips}, where instead of a list of independent triangles, you give a list of
consecutive triangles.  This needs much less space since (except for the
first triangle in the strip), you only need to specify one vertex per
triangle instead of three.  However, since converting between each
representation is straightforward (but somewhat cumbersome), we will use only
the simplest representation of meshes.

Besides the lists~$P$ and ~$T$ described above, some other lists can also be
computed.  For example, a list~$E$ of edges~$E=\{(p_i,q_i)\}\subset\N^2$
indicating which pairs of vertices are connected by an edge of a triangle.
Other lists that are sometimes useful and that can be easily computed on the
spot:
\begin{itemize}
	\item For each triangle, the list of its three edges
	\item For each edge, the list of its two neighboring triangles
	\item For each vertex, a list of its neighboring edges, traversed in
		anti-clockwise order.
	\item For each vertex, a list of its neighboring triangles, traversed in
		anti-clockwise order.
\end{itemize}
The information of all these lists is redundant with basic representation
using only~$P$ and~$T$, so they are almost never stocked in a file.  However,
sometimes we need to compute them because they allow for faster access (for
example, to compute the gradient of a function, you will need to know
neighboring points).

\subsection{Functions on discrete surfaces}

A function~$f:S\to\R$ defined on a surface is called a \emph{texture}.

Given a discretized surface, there are three straightforward ways to
associate a texture to it.

\begin{enumerate}
	\item Assign a number to each triangle ($f:T\to\R$)
	\item Assign a number to each vertex ($f:P\to\R$)
	\item Build an atlas of the surface, and define a regular 2D image
		over each chart in the atlas.  Check that the compatibility
		condition~\ref{eq:compat} is verified.
\end{enumerate}

The first two options are immediate to implement: just enlarge the list~$T$
or~$P$ with an additional column saying the value of each element (triangle
or vertex).  To implement the third option, we first need a way to represent
an atlas.  An easy way is to make a non-overlapping atlas: to each triangle,
we assign a single position in the plane.  This is achieved by enlarging the
list~$T$ with six additional columns, giving the position in the euclidean
plane of each of the vertices of the triangle.  Then, to define the
function~$f$ you simply give a 2D image in the euclidean plane.

Once you have defined a function in such a way, how do you evaluate it at a
point~$p\in S$?  On each of the three cases above:
\begin{enumerate}
	\item If~$p$ is in the interior of the triangle, give the value of
		this triangle.  If~$p$ is on an edge between two triangles,
		give the average of the values of each.  If~$p$ is a vertex,
		give a weighted combination of the neighboring triangles.
	\item $f(p)$ is a linear combination of the values of~$p$ on
		neighboring vertices, where the coefficients are the
		barycentric coordinates.
	\item To evaluate $f(p)$, project the point to the euclidean plane
		using the corresponding chart, and interpolate the discrete
		image that defines the texture: $f(p)=I_i(\varphi_i(p))$,
		where~$I_i$ is the discrete image on the euclidean plane of
		chart~$i$.
\end{enumerate}

In the third case, if~$p$ is on an edge between two triangles, the charts may
be different; but thanks to the compatibility condition the result will be
the same if we chose the triangle at either side of the edge.


This is the story for evaluating functions... but typically we want to do
other things with functions: combine them, differentiate them, solve PDE,
etc.  For the first two cases, discrete derivatives can be defined using
discrete exterior calculus (for example).  In the third case, they are
defined using the coordinate expressions applied to the textures of each
chart.  However, {\bf after each operation the compatibility condition must be
re-imposed so that the texture is consistent.}

\subsection{Charted versus pointwise functions}

It seems that all required operations are possible with either
representation.  So the choice of which one to use is mostly a matter of
convenience (runtime speed, stockage requirements, ease of implementation).
Bear in mind that our goal is to solve different PDE and optimisation on the
surface.

First of all, we discard the first representation of assigning a constant
value to each triangle.

Now, we compare the compromises between the other options: representing a
function as values on the vertices, or as a discrete image for each chart of
an atlas.

{\bf Advantages of atlases:}

\begin{itemize}
	\item It is the standard in modern computer graphics
	\item You can represent high-resolution textures with very few
		triangles (useful for flat areas)
	\item The textures of each chart are stored in regular png or tiff
		images
	\item There is a very clear separation between geometry and texture
		(once the atlas is computed)
	\item The same atlas may be used for different textures
	\item A bad mesh with a good texture will look good.  Thus, we can
		start working right away with whatever mesh and see \emph{the
		windows in the faˤades of the buildings}.
\end{itemize}

{\bf Disadvantages of atlases:}

\begin{itemize}
	\item You have to build the atlas from the mesh, which is a non-trivial
		operation
	\item For complex meshes, you may need an arbitrarily large number
		of different charts.  The number of required charts increases
		with the resolution of the geometry.
	\item If you produce a new texture, you have to enforce the
		compatibility conditions between the charts (otherwise the
		representation is not meaningful).  To enforce the
		compatibility condition you need to solve
		a system of Poisson equations, one for each chart, with
		coupled boundaries on the overlapping edges.  This
		compatibility must be enforced every time the texture data
		changes (e.g. on each iteration when solving a PDE
		iteratively).
	\item Computing derivatives (e.g. the laplacian) is not
		straightforward, as it requires computing differences
		between pixels interpolated on separate images.
	\item If you change the geometry (e.g., divide or join some
		triangles), you have to update the atlas, and charts may
		appear or disappear.
\end{itemize}

{\bf Advantages of vertex-wise functions:}

\begin{itemize}
	\item It is the standard in numerical analysis and simulation.
	\item It is straightforward. You just store the value of the function
		at each vertex of your surface (by e.g. adding a fourth
		column to the list of vertex positions).
	\item Any set of numbers is self-consistent, there are no
		compatibility conditions to impose
	\item Refining or joining mesh triangles is local and very easy: to add
		new vertices you interpolate the data using barycentric
		coordinates; to remove a vertex you sum his value into the
		neighboring ones (or you simply remove the vertex).
	\item Computing derivatives is easy, and only entails combination of
		neighboring values with weights given by the geometry
		(discrete first fundamental form).
	\item Discrete exterior calculus provides an appropriate
		formalization for PDE in this setting.
\end{itemize}

{\bf Disadvantages of vertex-wise functions:}

\begin{itemize}
	\item You need to use tiny triangles to represent high-resolution
		textures.  The size of the triangle must be smaller than the
		desired resolution.
	\item Even if the triangles are very small, this amounts to a
		re-sampling of the original image data that may create
		artifacts.
	\item Large meshes will be huge and sluggish to display on (e.g.
		meshlab)
	\item A bad mesh cannot ever give good visual results.  For example,
		in the case of buildings, if we have large triangles along
		the faˤade~\emph{we will never be able to put windows inside
		them}.
	\item ...Thus, it requires to refine the mesh as an essential, not
		purely cosmetic step.
\end{itemize}

Disclaimer (ENRIC): I am quite convinced that the vertex-wise representation
of functions is easier to work with for differential equations, and thus the
appropriate choice choice.  So take my comparison above with a grain of salt.

Note: maybe we can have the best of both worlds.  Since refining and
simplifying a mesh are standard and well-known operations, we can use the
atlases for storing a final, efficient mesh, and still keep the vertex-wise
representation for all the intermediate processing.  These operations are
standard in the CGAL library.

\end{otherlanguage}



%Either by giving a set
%
%
%of overlapping \emph{parametrizations}~$\varphi_i:\R^2\to\R^3$, or an by
%giving a set of \emph{charts}
%
%If~$S\subset\R^3$ is a surface, a function~$f:S\to\R$ is said to be smooth if
%it is the restriction to~$S$ of a smooth function of~$\R^3$.  Given a 








\clearpage
\section{Fusion criteria}

Our problem is now the following: produce a beautiful texture onto a given
mesh, obtained by taking the data from multiple images.  Thanks to the work
described above, we can assume that each point of the mesh can be mapped very
precisely into any of the images from where it is visible.  The final texture
will be obtained by combining all these colors in a consistent way.

The whole texturing pipeline has five steps:

\begin{enumerate}
	\item {\bf Preprocessing.} This is an optional step, where we may
		normalize/equalize the colors of the images so that they are
		easier to merge afterwards.  This involves also the
		pansharpening step if we want to work with color or
		multispectral images.
	\item {\bf Data projection.}
		Here we assign to each vertex of the mesh a list of
		features to merge.  In the simplest case it is just the pixel
		values projected from the images, but it can also be the
		gradients, drift-fields or other pointwise descriptors.
		We also compute and store all the required ancillary
		information, such as the angles between the surface normal
		and the point of view, the shadows (simulated from the
		geometry, or estimated by intensity thresholding), and the
		shadow trimaps.
	\item {\bf Data smoothing.}
		This is an optional step, where we may smooth-out the weight
		field, the shadow boundaries, filter obviously outlying data,
		etc.  We may also use the shadows to discard or weigh part of
		the data.
	\item {\bf Data fusion.}
		This is the core step, where the list of features assigned to
		each point is merged into a single feature for that point.
		This involves always taking a (possibly weighted)
		aggregation of the feature vectors, maybe with some pre and
		post processing (e.g., compute the gradients, merge them,
		then solve Poisson equation).
	\item {\bf Data post-processing.}
		Here we convert the feature vectors at each point into 8-bit RGB
		values.  It may involve some type of color normalization
		(retinex, enhance saturation).
\end{enumerate}

The core step is the~\emph{fusion}, where an aggregator is used to combine
several features into one.  An aggregator is a function~$f:\R^{d\times
N}\to\R^d$ that combines~$N$ features into a single one, where the features
are~$d$-dimensional vectors.

The simplest possible aggregator is the average function:
\[
	\mathrm{avg}(x_1,\ldots,x_N)=
	\frac{x_1+\cdots+x_N}{N}
\]
We will consider three families of aggregators that have the average function
as a particular case.  Each of these families depends on a real-valued
parameter~$p$:

The~\emph{power means} only make sense when all the features are positive
real numbers:
\[
	M_p(x_1,\ldots,x_N):=
	\sqrt[p]{ \frac{x_1^p+\cdots+x_N^p}{N} }
\]

The~\emph{Lehmer means}, or self-weighted means, work also for vector-valued features:
\[
	L_p(x_1,\ldots,x_N):=
	\frac
	{\|x_1\|^px_1+\cdots+\|x_N\|^px_N}
	{\|x_1\|^p+\cdots+\|x_N\|^p}
\]

And the~\emph{Fréchet means}, which minimize a power mean of the errors:
\[
	F_p(x_1,\ldots,x_N):=
	\mathrm{arg}\min_{m}\sum_{i=1}^N
	\left\|
	m-x_i
	\right\|^p
\]

Notice that~$\mathrm{avg}=M_1=L_0=F_2$.  Other particular cases
are~$L_{-\infty}=$~the vector of smallest norm, $L_{\infty}=$~the vector of
largest norm, $F_1=$~the geometric median, $F_{0^+}=$~the ``mode''.  The
aggregators~$M_p$ and~$L_p$ are straightforward to compute.  To compute~$F_p$
requires solving an optimization problem.  For~$p=2$ the solution is explicit
($F_2=M_1$), and for~$p=1$ there are very efficient algorithms to find it.
In any case, for~$p>=1$ the problem is smooth, convex and separable and a
couple of iterations of any reasonable iterative method will get to the
solution very fast.  As~$p$ goes to infinity,~$F_p$ converges to the midpoint
of the pair of features that are further apart.  For~$p<1$ the problem is not
necessarily convex and the solution is better approximated by using
heuristics.  For example, a ``soft mode''~$F_\epsilon$ can be found reliably
by the~$X$-medians algorithm.

Notice that different aggregators can be combined in~\emph{polar
coordinates}.  For example,~$\displaystyle \frac{F_1\|L_\infty\|}{\|F_1\|}$
gives a vector that has the same direction as the geometric median, and the
maximum length.

\bigskip

All the aggregators above can be~\emph{weighted} if you have a set
of~$N$ positive weights~$\omega_1,\ldots,\omega_N$.  To steer the importance
of these weights, we raise them to a power~$q\ge0$, where~$q$ is a new
parameter:
\[
	M_{p,q}(\omega_1,\ldots,\omega_N;\ x_1,\ldots,x_N):=
	\sqrt[p]{
	\frac{\omega_1^qx_1^p+\cdots+\omega_N^qx_N^p}
	{\omega_1^q+\cdots+\omega_N^q} }
\]
\[
	L_{p,q}(\omega_1,\ldots,\omega_N;\ x_1,\ldots,x_N):=
	\frac
	{\omega_1^q\|x_1\|^px_1+\cdots+\omega_N^q\|x_N\|^px_N}
	{\omega_1^q\|x_1\|^p+\cdots+\omega_N^q\|x_N\|^p}
\]
\[
	F_{p,q}(\omega_1,\ldots,\omega_N;\ x_1,\ldots,x_N):=
	\mathrm{arg}\min_{m}\sum_{i=1}^N
	\omega_i^q
	\left\|
	m-x_i
	\right\|^p
\]
In all the cases above, setting~$q=0$ reduces to the unweighted case.  Also,
setting~$q=\infty$ reduces to selecting the feature with the largest
corresponding weight.  The weight formalism is useful to reject some points
from the aggregation, by setting their weight to zero.  Notice that in that
case we use the convention~$0^0=1$.

More importantly, the zero weights can be used to reject whole *images* from the
collection.  This simplifies the scripting considerably: you always compute
the fusion of all the images, but set some weight to zero if there is an
image that you want to discard.  In what follows we assume that the
weight~$w_i$ of a point is the positive part of the cosine of the angle between the normal at the
surface on that point and the line of sight
\[
	\omega_i = \max\left(0,\cos\left(\vec n,
	\mathrm{sight}_i\right)\right).
\]
Thus the weight is~$1$ for exactly fronto-parallel surface patches, and
decreases until 0 for perpendicular and invisible surface patches.
Furthermore, the weight can be set to zero depending on the shadow trimaps
and the fusion criterion used.

Here are some examples of fusion criteria (without using the shadow trimaps):

\begin{enumerate}
	\item {\bf Naive fusion~$L_{0,0}$.}  You just compute the unweighted
		average of the input colors.
	\item {\bf Most frontal face~$L_{0,\infty}$.}  You select the color
		of the image that is most frontal at each point.
	\item {\bf Robust fusion~$F_{1,0}$.}  The unweighted geometric median
		of all the colors.
	\item {\bf Weighted robust fusion~$F_{1,2}$.}  The weighted geometric
		median of all the colors.
	\item {\bf More robust fusion~$F_{\tfrac12,0}$.}  The soft mode of all
		the colors (estimated by x-medians).
	\item {\bf More robust weighted fusion~$F_{\tfrac12,2}.$}  The
		weighted soft mode of all the colors
	\item {\bf PDE-based fusion.}  Let~$D$ be a differential operator
		and~$f$ an aggregator function.  You compute~$D^{-1}\circ
		f\circ D$.  Notice that~$D^{-1}$ involves solving a PDE, and
		you typically have to give a Dirichlet boundary condition on
		some points (always), and optionally a few Neumann boundary
		conditions.  Typical examples include~$D=$~gradient
		and~$D^{-1}=$~Poisson equation, or~$D=$~drift field
		and~$D^{-1}=$~osmosis equation.  For~$D=D^{-1}=$~identity we
		recover the previous cases based on color values.
		Thus, this case is the most general one.
\end{enumerate}

The fusion by PDE is very important.  Of fundamental importance is the
observation that {\bf the choices of~$f$ and~$D$ are completely independent}.
For example, suppose that we consider only the agregators~$L_0$ and~$F_1$,
the three values of~$q=0,2,\infty$, and the three choices of~$D$ (identity,
gradient or drift field).  This gives a total of at least 18 fusion criteria
to try.

\bigskip

Next: enumeration of different ways to use the shadow trimaps~$T_i$.  Let us
recall that~$T_i$ takes values between 0 (completely inside the shadow), 1
completely outside the shadow) and interpolates depending to our belief 

\begin{enumerate}
	\item Ignore the shadow trimaps
	\item Multiply the weights of the aggregator by 0, at the points
		where the trimap does not equal to 1.  Thus, the features
		that are inside the shadow are not used.
	\item Multiply the weights of the aggregator by the trimaps values.
		If the trimap is smooth, this allows a smooth transition
		between the inside and the outside of the shadow.
	\item Set the drift field to zero where the trimap does not equal 0
		or 1.  Thus, only drift fields that are completely inside or
		completely outside the shadow are used.
	\item Solve the PDE on the region~$T<1$, with Dirichlet boundary
		condition determined by aggregating the images on the
		region~$T=1$.
\end{enumerate}


\end{document}
% vim: set tw=77 spell spelllang=en:

