\documentclass{article}
\usepackage[utf8]{inputenc}         % to type accents
\usepackage[T1]{fontenc}            % for automatic french-style guillemets
\usepackage[british,french]{babel}  % french, english typographical conventions
\usepackage{amssymb}                % fancy symbols like \checkmark
\usepackage{color}                  % colorized text
\usepackage{url,hyperref}           % to make clickable hyperlinks
\usepackage{mathptmx}               % times new roman font

% macros for math
\def\R{\mathbf{R}} % real numbers
\def\N{\mathbf{N}} % natural numbers

% macros for colorized comments
\newcommand{\mnhrdt}[1]{{\footnotesize\textcolor[rgb]{0.8,0.5,0.5}{eml: #1}}}

% smaller margins
\addtolength{\hoffset}{-3em}
\addtolength{\voffset}{-3em}
\addtolength{\textwidth}{6em}
\addtolength{\textheight}{6em}

\title{Avancement sur la texturation}
\author{Marie d'Autume}


\begin{document}
\maketitle

%%%%%%%%%%%%%%%%%
%%%%TO DO LIST%%%%
%%%%%%%%%%%%%%%%%
\section{Liste de travaux à faire}
\label{sec:todolist}

\begin{itemize}
\item[\checkmark] Création d'un mesh grossier à partir du lidar, coloré à l'aide d'une image : \verb"colorsingle.c"
\item[\checkmark] Création d'un mesh grossier à partir du lidar, coloré à partir de plusieurs images, choix de l'image la plus en face : \verb"colormultiple.c"
\item z-buffer prenant en comptes les triangles, pas simplement les points du lidar
\item création d'un mesh le plus propre possible
	\mnhrdt{C'est important, mais il ne faut pas y perdre trop de temps
		maintenant, car c'est un sujet qui commute avec la
		texturisation.  Si le filtrage par courbure fonctionne, on le
		laisse pour l'instant.}
\begin{itemize}
\item[\checkmark] régulariser le lidar (fait à l'aide du curvature microscope
sur ipol~\footnote{\url{http://www.ipol.im/pub/pre/212/}})
\item[\checkmark]  créer les faces triangulaires selon la hauteur des sommets et
	non toujours pareil
		\mnhrdt{C'est un critère facile à implementer (il faut
		simplement choisir l'arête diagonale la plus courte).
		Cependant, ceci ne va pas résoudre le problème des façades
		non-verticales.}
\item (19/07) regarder ``smoothing ... bilateral'' de Julie Digne sur ipol pour
régulariser les normales des façades.
		\mnhrdt{Pourquoi? C'est un algorithme qui travaille avec des
		nuages de points, pas avec des maillages.  Je ne vois pas
		comment l'utiliser ici...} C'est Gabriele qui m'en a parl\'e, en compl\'ement du lissage de lidar et du d\'ecoupage de triangle pour les zones qui ne sont pas verticales ou horizontales.
\end{itemize}
\item (19/07) recaler les images en utilisant les données de gabriele sur menthe \url{menthe:/home/facciolo/iarpa/all_pairs/outdir_from_19_32/ncc_transform.txt"}. Exemple de mauvais recalage : Avec comme input le lidar r\'egularis\'e et les 9 premi\`eres images, colormultiple.c utilise la 5\`eme image pour l'arri\`ere de nombreux b\^atiments. Mais cette image est mal recal\'ee et on voit que les fa\c cades des b\^atiments sont \`a moiti\'e blanches (venant de zones satur\'ees de l'image correspondant aux toits des b\^atiments et non \`a leurs fa\c cades).
	\mnhrdt{Très important!  Il suffit de passer les vecteurs de recalage
		comme offset à appliquer à chaque image avant d'évaluer ses
		RPC.  Mais, comme la filtrage du lidar, l'implementation de
		cette opération commute avec la texturisation, et on peut la
		laisser pour la fin.}
\item (22/07) appliquer non local pansharpening aux images multispectrales pour obtenir des meshs en couleurs (JMM).

\item (19/07) STRUCTURES DE DONNÉES POUR LA FUSION DE TEXTURES.  \mnhrdt{L'objectif de ce
travail est avoir un maillage avec une texture nouvelle qui n'existe
pas encore (donc, l'atlas avec toutes les images d'entrée est
juste une étape intermédiaire qui ne fait pas partie de la
sortie).  Il faut d'ores et déjà avoir un algorithme (même
naïf) qui crée un mesh de sortie avec une seule texture
obtenue par fusion des toutes les images d'entrée.  On devra
essayer plusieurs critères de fusion, et pouvoir les comparer
facilement sur des nouvelles donnés d'entrée.  Notamment
\begin{itemize}
	\item À chaque triangle, la texture de l'image visible la plus frontale
	\item À chaque triangle, la texture de l'image visible de plus haute
		résolution une fois projetée sur le triangle.
	\item À chaque triangle, la moyenne arithmétique de toutes les
		images visibles.
	\item À chaque triangle, la médiane de toutes les images visibles.
	\item À chaque triangle, la moyenne de toutes les images pondérée par
		une fonction monotone de l'angle de visibilité de chacune.
	\item Comme avant, mais avec médiane pondérée
	\item Autres critère de fusion de couleurs, ou fusion pondérée (e.g.,
		modes, k-medians, etc).
	\item À chaque triangle, la moyenne des gradients de toutes les
		images visibles.  Puis résoudre l'équation de Poisson sur la
		surface entière pour récuperer une texture.
	\item À chaque triangle, la médiane (ou d'autres fonctions robustes,
		pondérées ou non,  des gradients de toutes les images
		visibles).  Puis résoudre l'équation de Poisson.
	\item Comme avant, mais en combinant drift-fields et utilisant
		l'équation d'Osmose sur la surface.
	\item Tous les critères antérieurs, mais utilisant un changement de
		contraste~$I_i\mapsto\alpha_i I_i+\beta_i$ sur chaque image,
		où les champs de coefficients $\alpha$ et $\beta$ sont très
		lisses et utilisées pour harmoniser les couleurs le plus
		possible avant de la fusion.  (Ce critère est nouveau, par
		rapport aux articles antérieurs, il me parait)
\end{itemize}
Il me parait que ce point est le plus important.  Avant de corriger des
résultats qui ne sont pas ``satisfaisants'' (c'est un peu subjectif), il faut
avoir un programme qui résout le problème de bout à bout, même s'il le fait
très mal.  Au moins, il faut implémenter deux ou trois critères de fusion de
textures et pouvoir comparer les ply texturisées de sortie.  Ceci est le plus
urgent.  Ensuite, on pourra améliorer la qualités des maillages, du recalage
des images, etc.
}
\end{itemize}

 %%%%%%%%%%%%%%%%%
%%%%%%%REMARQUES%%%%%%%
 %%%%%%%%%%%%%%%%%
\section{Remarque}
\subsection{Création de mesh à partir de lidar}
Le lidar est bruité, le bord des immeubles n'est pas lisse. Le mesh naïf
obtenu en reliant les points du lidar n'est donc pas satisfaisant pour du prototypage.
L'utilisation de ``The Image Curvature Microscope'' sur ipol permet d'obtenir
un lidar plus lisse à partir duquel on obtient un mesh plus propre. Plutôt
que de créer les triangles toujours dans le même ordre il faudrait aussi que
cela dépende de la hauteur des quatre points utilisés. Cela donnerait des
surfaces plus lisses pour les immeubles en diagonal sur le lidar.
\mnhrdt{Je ne suis pas d'accord que le mesh naïf n'est pas satisfaisant, le
problème est que on le regarde de trop près.  On peut le filtrer sans soucis
pour obtenir des maillages un peu plus faciles à voir. }

\subsection{R\'eponse aux commentaires sur la non n\'ecessit\'e d'avoir un beau mesh}
Quand j'ai parl\'e de mesh ``satisfaisant'' c'\'etait au sens de satisfaisant pour du prototypage et non comme r\'esultat final. Donc l'astuce de lisser le lidar et de cr\'eer les faces triangulaires selon la hauteur des sommets r\'esout pour l'instant le probl\`eme. Avec le lidar non liss\'e, les r\'esultats sont tr\`es difficiles \`a interpr\'eter et on ne peut pas juger la texturation. A fortiori lorsque les images ne sont pas recal\'es. On obtient de microscopiques morceaux de texture juxtapos\'es. Du coup am\'eliorer le mesh ne commute pas vraiment avec la texturation.

Je t'ai mis les r\'esultats sur menthe dans essai\_small pour lidar non filtr\'e et dans essai\_curve pour lidar filtr\'e et triangles d\'ecoup\'es selon la hauteur. 


 %%%%%%%%%%%%%%%%%
%%%%%%%AVANCEMENT%%%%%%%
 %%%%%%%%%%%%%%%%%
\section{Résumé des activités de la journée}
\paragraph{19/07/2017} création de ce pdf, essai de \verb"colormultiple.c" sur le lidar régularisé, d\'ecoupage plus intelligent de face triangulaire. \c Ca marche, on devine des fen\^etres ! travail en cours sur zbuffer.c (aml\'elioration du z-buffer de colorize\_with\_shadows.c)
\paragraph{20/07/2017}  travail toujours en cours sur zbuffer.c (j'ai pass\'e la journ\'ee \`a faire du debuggage et \`a am\'eliorer ma connaissance des pointeurs). zbuffer.c est plus ou moins fonctionnel. 
\paragraph{21/07/2017} nettoyage du main de colormultiple pour le s\'eparer en sous ensembles. D\'ebut de travail sur le recalage.
\paragraph{24/07/2017} recalage en cours (j'esp\`ere avoir des r\'esultats demain mais cela m'\'etonnerait qu'ils soient bons : les vecteurs de d\'eplacement semblent trop \'elev\'es). \'etude de pansharpening pendant que la grande image ntf charge.

\clearpage
\begin{otherlanguage}{british}
\section{Data structures for surfaces, and for functions on surfaces}

\subsection{Continuous setting}

A surface is a smooth 2-dimensional sub-manifold of~$\R^3$.
There are three common ways to define a surface~$S$:
\begin{itemize}
	\item \emph{Implicit} representation by a function~$f:\R^3\to\R$,
		so that~$S=f^{-1}(\{0\})$.
	\item \emph{Parametric} representation by a function~$s:\R^2\to\R$,
		so that~$S=s(\R^2)$.

	\item By a set of \emph{coordinate charts} $(U_i,\varphi_i)$,
		where~$S=\bigcup U_i$, $\varphi_i:U_i\to\R^2$ and the transition
		functions~$\varphi_i\circ\varphi_j^{-1}:\R^2\to\R^2$ are
		smooth.  A set of coordinate charts is called
		an~\emph{atlas}.
\end{itemize}
The implicit and atlas representations allow for arbitrary topologies.  The
parametric representation is the simplest one, but it can only represent
surfaces homeomorphic to a square.  An even more restricted parametric
representation is the 2.5D representation, where the parametrization $s(x,y)$
is of the form~$s(x,y)=(x,y,z(x,y))$ where~$z:\R^2\to\R$.

When working with surfaces, we usually work with functions~$f:S\to\R$.
There are different ways to define a function~$f:S\to\R$.  One possibility is
to define a function $g:\R^3\to\R$, and then set~$f=g|_S$.  However, this
representation is non-unique, since many different~$g$ can give the same~$f$.
Another possibility, if~$S$ is defined by an atlas~$\{U_i,\varphi_i\}$, is to
give the functions~$f_i=f\circ\varphi_i^{-1}:\R^2\to\R$.
%If~$S\subset\R^3$ is a surface, a function~$f:S\to\R$ is said to be smooth
%when it is the restriction to~$S$ of a smooth function of~$\R^3$.  To check
%that a function is smooth (or other properties of the function), it depends
%on how the surface is defined (by parametrization or by atlas).
%\begin{itemize}
%	\item If~$S$ is defined by a parametrization~$s:\R^2\to\R^3$, we need
%		to check that the function~$f\circ s:\R^2\to\R$ is smooth.
%	\item If~$S$ is defined by an atlas~$\{U_i,\varphi_i\}$, we need to
%		check that the functions~$f\circ\varphi_i^{-1}:\R^2\to\R$ are
%		smooth.
%\end{itemixe}
Notice, however, that an arbitrary collection of functions~$f_i:\R^2\to\R$
does~\emph{not} in general define a function on~$S$ via this atlas.  A
compatibility criterion is required, whereby
\begin{equation}\label{eq:compat}
f_i\circ\varphi_i=f_j\circ\varphi_j\quad\textrm{when}\quad U_i\cap U_j\neq\emptyset
\end{equation}
		(the
local definitions the function must be compatible on overlapping charts).

A standard construction in differential geometry is the Laplace-Beltrami
operator~$\Delta_g$ on a surface~$S$, which is a generalization of the Laplacian
operator~$\Delta=\frac{\partial^2}{\partial x^2}+\frac{\partial^2}{\partial
y^2}$ in the euclidean plane.  It allows to state, for example, Poisson
equation on the surface
\[
	\Delta_g u = f
\]
that has a unique solution up to an additive constant.

In a coordinate chart, the Laplace-Beltrami operator has the
form~$\alpha\mathrm{div}\left(A\nabla u\right)$.  This is an anisotropic
diffusion operator where the matrix field~$A$ and scalar field~$\alpha$ are
formed by the coefficients of the first fundamental form of the surface
(derivatives of the parametrization).

Similarly, you can define gradients and divergences for functions defined on
the surface, and state Osmosis equation, and so on.


\subsection{Discretization of a surface}

The standard way to discretize a surface is to use a mesh of
triangles~\footnote{simple or low-resolution surfaces can also be represented
very efficiently by implicit functions, and this allows for very fast
computer graphics operations.  However, it does not scale well when you need a
lot of geometric detail, so we will ignore this technique here}.
A mesh of~$n$ vertices and~$m$ triangles is represented by two lists:
\begin{enumerate}
	\item A list of~$n$ points~$P=\{(x_i,y_i,z_i)\}_{i=1,\ldots,
		n}\subset\R^3$ giving the spatial positions of the vertices.
	\item A list of~$m$ triangles~$T=\{(a_j,b_j,c_j)\}_{j=1,\ldots,
		m}\subset\N^3$ giving the indices of the vertices of each
		triangle (as positions inside the list~$P$).
\end{enumerate}

There are also other, fancier data structures, for example~\emph{triangle
strips}, where instead of a list of independent triangles, you give a list of
consecutive triangles.  This needs much less space since (except for the
first triangle in the strip), you only need to specify one vertex per
triangle instead of three.  However, since converting between each
representation is straightforward (but somewhat cumbersome), we will use only
the simplest representation of meshes.

Besides the lists~$P$ and ~$T$ described above, some other lists can also be
computed.  For example, a list~$E$ of edges~$E=\{(p_i,q_i)\}\subset\N^2$
indicating which pairs of vertices are connected by an edge of a triangle.
Other lists that are sometimes useful and that can be easily computed on the
spot:
\begin{itemize}
	\item For each triangle, the list of its three edges
	\item For each edge, the list of its two neighboring triangles
	\item For each vertex, a list of its neighboring edges, traversed in
		anti-clockwise order.
	\item For each vertex, a list of its neighboring triangles, traversed in
		anti-clockwise order.
\end{itemize}
The information of all these lists is redundant with basic representation
using only~$P$ and~$T$, so they are almost never stocked in a file.  However,
sometimes we need to compute them because they allow for faster access (for
example, to compute the gradient of a function, you will need to know
neighboring points).

\subsection{Functions on discrete surfaces}

A function~$f:S\to\R$ defined on a surface is called a \emph{texture}.

Given a discretized surface, there are three straightforward ways to
associate a texture to it.

\begin{enumerate}
	\item Assign a number to each triangle ($f:T\to\R$)
	\item Assign a number to each vertex ($f:P\to\R$)
	\item Build an atlas of the surface, and define a regular 2D image
		over each chart in the atlas.  Check that the compatibility
		condition~\ref{eq:compat} is verified.
\end{enumerate}

The first two options are immediate to implement: just enlarge the list~$T$
or~$P$ with an additional column saying the value of each element (triangle
or vertex).  To implement the third option, we first need a way to represent
an atlas.  An easy way is to make a non-overlapping atlas: to each triangle,
we assign a single position in the plane.  This is achieved by enlarging the
list~$T$ with six additional columns, giving the position in the euclidean
plane of each of the vertices of the triangle.  Then, to define the
function~$f$ you simply give a 2D image in the euclidean plane.

Once you have defined a function in such a way, how do you evaluate it at a
point~$p\in S$?  On each of the three cases above:
\begin{enumerate}
	\item If~$p$ is in the interior of the triangle, give the value of
		this triangle.  If~$p$ is on an edge between two triangles,
		give the average of the values of each.  If~$p$ is a vertex,
		give a weighted combination of the neighboring triangles.
	\item $f(p)$ is a linear combination of the values of~$p$ on
		neighboring vertices, where the coefficients are the
		barycentric coordinates.
	\item To evaluate $f(p)$, project the point to the euclidean plane
		using the corresponding chart, and interpolate the discrete
		image that defines the texture: $f(p)=I_i(\varphi_i(p))$,
		where~$I_i$ is the discrete image on the euclidean plane of
		chart~$i$.
\end{enumerate}

In the third case, if~$p$ is on an edge between two triangles, the charts may
be different; but thanks to the compatibility condition the result will be
the same if we chose the triangle at either side of the edge.


This is the story for evaluating functions... but typically we want to do
other things with functions: combine them, differentiate them, solve PDE,
etc.  For the first two cases, discrete derivatives can be defined using
discrete exterior calculus (for example).  In the third case, they are
defined using the coordinate expressions applied to the textures of each
chart.  However, {\bf after each operation the compatibility condition must be
re-imposed so that the texture is consistent.}

\subsection{Charted versus pointwise functions}

It seems that all required operations are possible with either
representation.  So the choice of which one to use is mostly a matter of
convenience (runtime speed, stockage requirements, ease of implementation).
Bear in mind that our goal is to solve different PDE and optimisation on the
surface.

First of all, we discard the first representation of assigning a constant
value to each triangle.

Now, we compare the compromises between the other options: representing a
function as values on the vertices, or as a discrete image for each chart of
an atlas.

{\bf Advantages of atlases:}

\begin{itemize}
	\item It is the standard in modern computer graphics
	\item You can represent high-resolution textures with very few
		triangles (useful for flat areas)
	\item The textures of each chart are stored in regular png or tiff
		images
	\item There is a very clear separation between geometry and texture
		(once the atlas is computed)
	\item The same atlas may be used for different textures
	\item A bad mesh with a good texture will look good.  Thus, we can
		start working right away with whatever mesh and see \emph{the
		windows in the façades of the buildings}.
\end{itemize}

{\bf Disadvantages of atlases:}

\begin{itemize}
	\item You have to build the atlas from the mesh, which is a non-trivial
		operation
	\item For complex meshes, you may need an arbitrarily large number
		of different charts.  The number of required charts increases
		with the resolution of the geometry.
	\item If you produce a new texture, you have to enforce the
		compatibility conditions between the charts (otherwise the
		representation is not meaningful).  To enforce the
		compatibility condition you need to solve
		a system of Poisson equations, one for each chart, with
		coupled boundaries on the overlapping edges.  This
		compatibility must be enforced every time the texture data
		changes (e.g. on each iteration when solving a PDE
		iteratively).
	\item Computing derivatives (e.g. the laplacian) is not
		straightforward, as it requires computing differences
		between pixels interpolated on separate images.
	\item If you change the geometry (e.g., divide or join some
		triangles), you have to update the atlas, and charts may
		appear or disappear.
\end{itemize}

{\bf Advantages of vertex-wise functions:}

\begin{itemize}
	\item It is the standard in numerical analysis and simulation.
	\item It is straightforward. You just store the value of the function
		at each vertex of your surface (by e.g. adding a fourth
		column to the list of vertex positions).
	\item Any set of numbers is self-consistent, there are no
		compatibility conditions to impose
	\item Refining or joining mesh triangles is local and very easy: to add
		new vertices you interpolate the data using barycentric
		coordinates; to remove a vertex you sum his value into the
		neighboring ones (or you simply remove the vertex).
	\item Computing derivatives is easy, and only entails combination of
		neighboring values with weights given by the geometry
		(discrete first fundamental form).
	\item Discrete exterior calculus provides an appropriate
		formalization for PDE in this setting.
\end{itemize}

{\bf Disadvantages of vertex-wise functions:}

\begin{itemize}
	\item You need to use tiny triangles to represent high-resolution
		textures.  The size of the triangle must be smaller than the
		desired resolution.
	\item Even if the triangles are very small, this amounts to a
		re-sampling of the original image data that may create
		artifacts.
	\item Large meshes will be huge and sluggish to display on (e.g.
		meshlab)
	\item A bad mesh cannot ever give good visual results.  For example,
		in the case of buildings, if we have large triangles along
		the façade~\emph{we will never be able to put windows inside
		them}.
	\item ...Thus, it requires to refine the mesh as an essential, not
		purely cosmetic step.
\end{itemize}

Disclaimer (ENRIC): I am quite convinced that the vertex-wise representation
of functions is easier to work with for differential equations, and thus the
appropriate choice choice.  So take my comparison above with a grain of salt.

Note: maybe we can have the best of both worlds.  Since refining and
simplifying a mesh are standard and well-known operations, we can use the
atlases for storing a final, efficient mesh, and still keep the vertex-wise
representation for all the intermediate processing.  These operations are
standard in the CGAL library.

\end{otherlanguage}



%Either by giving a set
%
%
%of overlapping \emph{parametrizations}~$\varphi_i:\R^2\to\R^3$, or an by
%giving a set of \emph{charts}
%
%If~$S\subset\R^3$ is a surface, a function~$f:S\to\R$ is said to be smooth if
%it is the restriction to~$S$ of a smooth function of~$\R^3$.  Given a 






\end{document}
% vim: set tw=77 spell spelllang=fr:

