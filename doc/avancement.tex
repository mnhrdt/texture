\documentclass{article}
\usepackage[utf8]{inputenc}    % to type accents
\usepackage[T1]{fontenc}       % for automatic french-style guillemets
\usepackage[french]{babel}     % french typographical conventions
\usepackage{amssymb}           % fancy symbols like \checkmark
\usepackage{color}             % colorized text
\usepackage{url,hyperref}      % to make clickable hyperlinks
\usepackage{mathptmx}          % times new roman font

% macros for colorized comments
\newcommand{\mnhrdt}[1]{{\footnotesize\textcolor[rgb]{0.8,0.5,0.5}{eml: #1}}}

% smaller margins
\addtolength{\hoffset}{-2em}
\addtolength{\voffset}{-2em}
\addtolength{\textwidth}{4em}
\addtolength{\textheight}{4em}

\title{Avancement sur la texturation}
\author{Marie d'Autume}


\begin{document}
\maketitle

%%%%%%%%%%%%%%%%%
%%%%TO DO LIST%%%%
%%%%%%%%%%%%%%%%%
\section{Liste de travaux à faire}
\label{sec:todolist}

\begin{itemize}
\item[\checkmark] Création d'un mesh grossier à partir du lidar, coloré à l'aide d'une image : \verb"colorsingle.c"
\item[\checkmark] Création d'un mesh grossier à partir du lidar, coloré à partir de plusieurs images, choix de l'image la plus en face : \verb"colormultiple.c"
\item z-buffer prenant en comptes les triangles, pas simplement les points du lidar
\item création d'un mesh le plus propre possible
	\mnhrdt{C'est important, mais il ne faut pas y perdre trop de temps
		maintenant, car c'est un sujet qui commute avec la
		texturisation.  Si le filtrage par courbure fonctionne, on le
		laisse pour l'instant.}
\begin{itemize}
\item[\checkmark] régulariser le lidar (fait à l'aide du curvature microscope
sur ipol~\footnote{\url{http://www.ipol.im/pub/pre/212/}})
\item (19/07) créer les faces triangulaires selon la hauteur des sommets et
	non toujours pareil
		\mnhrdt{C'est un critère facile à implementer (il faut
		simplement choisir l'arête diagonale la plus courte).
		Cependant, ceci ne va pas résoudre le problème des façades
		non-verticales.}
\item (19/07) regarder ``smoothing ... bilateral'' de Julie Digne sur ipol pour
régulariser les normales des façades.
		\mnhrdt{Pourquoi? C'est un algorithme qui travaille avec des
		nuages de points, pas avec des maillages.  Je ne vois pas
		comment l'utiliser ici...}
\end{itemize}
\item (19/07) recaler les images en utilisant les données de gabriele sur menthe \url{menthe:/home/facciolo/iarpa/all_pairs/outdir_from_19_32/ncc_transform.txt"}
	\mnhrdt{Très important!  Il suffit de passer les vecteurs de recalage
		comme offset à appliquer à chaque image avant d'évaluer ses
		RPC.  Mais, comme la filtrage du lidar, l'implementation de
		cette opération commute avec la texturisation, et on peut la
		laisser pour la fin.}
		\mnhrdt{
\item STRUCTURES DE DONNÉES POUR LA FUSION DE TEXTURES.  L'objectif de ce
travail est avoir un maillage avec une texture nouvelle qui n'existe
pas encore (donc, l'atlas avec toutes les images d'entrée est
juste une étape intermédiaire qui ne fait pas partie de la
sortie).  Il faut d'ores et déjà avoir un algorithme (même
naïf) qui crée un mesh de sortie avec une seule texture
obtenue par fusion des toutes les images d'entrée.  On devra
essayer plusieurs critères de fusion, et pouvoir les comparer
facilement sur des nouvelles donnés d'entrée.  Notamment
\begin{itemize}
	\item À chaque triangle, la texture de l'image visible la plus frontale
	\item À chaque triangle, la texture de l'image visible de plus haute
		résolution une fois projetée sur le triangle.
	\item À chaque triangle, la moyenne arithmétique de toutes les
		images visibles.
	\item À chaque triangle, la médiane de toutes les images visibles.
	\item À chaque triangle, la moyenne de toutes les images pondérée par
		une fonction monotone de l'angle de visibilité de chacune.
	\item Comme avant, mais avec médiane pondérée
	\item Autres critère de fusion de couleurs, ou fusion pondérée (e.g.,
		modes, k-medians, etc).
	\item À chaque triangle, la moyenne des gradients de toutes les
		images visibles.  Puis résoudre l'équation de Poisson sur la
		surface entière pour récuperer une texture.
	\item À chaque triangle, la médiane (ou d'autres fonctions robustes,
		pondérées ou non,  des gradients de toutes les images
		visibles).  Puis résoudre l'équation de Poisson.
	\item Comme avant, mais en combinant drift-fields et utilisant
		l'équation d'Osmose sur la surface.
	\item Tous les critères antérieurs, mais utilisant un changement de
		contraste~$I_i\mapsto\alpha_i I_i+\beta_i$ sur chaque image,
		où les champs de coefficients $\alpha$ et $\beta$ sont très
		lisses et utilisées pour harmoniser les couleurs le plus
		possible avant de la fusion.  (Ce critère est nouveau, par
		rapport aux articles antérieurs, il me parait)
\end{itemize}
Il me parait que ce point est le plus important.  Avant de corriger des
résultats qui ne sont pas ``satisfaisants'' (c'est un peu subjectif), il faut
avoir un programme qui résout le problème de bout à bout, même s'il le fait
très mal.  Au moins, il faut implémenter deux ou trois critères de fusion de
textures et pouvoir comparer les ply texturisées de sortie.  Ceci est le plus
urgent.  Ensuite, on pourra améliorer la qualités des maillages, du recalage
des images, etc.
}
\end{itemize}

 %%%%%%%%%%%%%%%%%
%%%%%%%REMARQUES%%%%%%%
 %%%%%%%%%%%%%%%%%
\section{Remarque}
\subsection{Création de mesh à partir de lidar}
Le lidar est bruité, le bord des immeubles n'est pas lisse. Le mesh naïf
obtenu en reliant les points du lidar n'est donc pas satisfaisant.
L'utilisation de ``The Image Curvature Microscope'' sur ipol permet d'obtenir
un lidar plus lisse à partir duquel on obtient un mesh plus propre. Plutôt
que de créer les triangles toujours dans le même ordre il faudrait aussi que
cela dépende de la hauteur des quatre points utilisés. Cela donnerait des
surfaces plus lisses pour les immeubles en diagonal sur le lidar.
\mnhrdt{Je ne suis pas d'accord que le mesh naïf n'est pas satisfaisant, le
problème est que on le regarde de trop près.  On peut le filtrer sans soucis
pour obtenir des maillages un peu plus faciles à voir.}


 %%%%%%%%%%%%%%%%%
%%%%%%%AVANCEMENT%%%%%%%
 %%%%%%%%%%%%%%%%%
\section{Résumé des activités de la journée}
\paragraph{19/07/2017} création de ce pdf, essai de \verb"colormultiple.c" sur le lidar régularisé







\end{document}
% vim: set tw=77 spell spelllang=fr:

