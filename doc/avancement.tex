\documentclass[12pt]{article}

%% Language and font encodings
\usepackage[latin1]{inputenc}
\usepackage{nicefrac}
\usepackage{color}
\usepackage{bm}
\usepackage{amssymb}
\usepackage{amsmath}
\usepackage{graphicx}
\usepackage{dsfont}
\usepackage{siunitx}
\usepackage[vlined,algoruled,linesnumbered]{algorithm2e}
\usepackage{breqn}
\usepackage{pdfsync}
\usepackage{multirow}
\usepackage{subfigure}
\usepackage{float}
\usepackage{chngcntr}
\usepackage{apptools}
\usepackage{algpseudocode}
\AtAppendix{\counterwithin{theorem}{section}}

% comments
\newcommand{\nota}[1]{\textcolor{blue}{\textbf{#1}}}
\newcommand{\add}[1]{\textcolor{blue}{\textit{#1}}}
\newcommand{\remove}[1]{\textcolor{red}{\textit{#1}}}


% algorithm
\algnewcommand\algorithmicswitch{\textbf{switch}}
\algnewcommand\algorithmiccase{\textbf{case}}
\algnewcommand\algorithmicassert{\texttt{assert}}
\algnewcommand\Assert[1]{\State \algorithmicassert(#1)}%
\SetKwComment{Comment}{}{}
\newcommand{\mycmtsty}[1]{\color{blue}\small\tt #1}
\SetCommentSty{mycmtsty}

% New "environments"
\algdef{SE}[SWITCH]{Switch}{EndSwitch}[1]{\algorithmicswitch\ #1\ \algorithmicdo}{\algorithmicend\ \algorithmicswitch}%
\algdef{SE}[CASE]{Case}{EndCase}[1]{\algorithmiccase\ #1}{\algorithmicend\ \algorithmiccase}%
\algtext*{EndSwitch}%
\algtext*{EndCase}%


% notations
\def\b{\textnormal{\bf{b}}}
\def\p{\textnormal{\bf{p}}}
\def\x{\textnormal{\bf{x}}}
\def\u{\textnormal{\bf{u}}}
\def\d{\textnormal{\bf{d}}}
\def\f{\mathbf{f}}
\def\X{\textnormal{\bf{\small X}}}
\def\C{\textnormal{\bf{\small C}}}
\def\S{\textnormal{\bf{\small S}}}
\def\E{\textnormal{\bf{\small E}}}
\def\Xc{\textnormal{\small X}}
\def\Yc{\textnormal{\small Y}}
\def\Zc{\textnormal{\small Z}}
\def\A{\mathbf{A}}
\def\P{\texttt{P}}
\def\F{\texttt{F}}
\def\H{\texttt{H}}
\def\ie{\textit{i.e.} }
\def\eg{\textit{e.g.} }
\def\etal{\textit{et al}. }
\DeclareMathOperator*{\argmin}{arg\,min}
\def\R{\mathbb{R}}
\def\N{\mathbf{N}}
\def\Z{\mathbf{Z}}
\def\div{\mathrm{div}}
\def\ind{\mathds{1}}



\title{Avancement sur la texturation}
\author{Marie d'Autume}

\begin{document}
\maketitle
%%%%%%%%%%%%%%%%%
%%%%TO DO LIST%%%%
%%%%%%%%%%%%%%%%%
\section{Liste de travaux � faire}
\label{sec:todolist}
\begin{itemize}
\item[\checkmark] Cr�ation d'un mesh grossier � partir du lidar, color� � l'aide d'une image : \verb"colorsingle.c"
\item[\checkmark] Cr�ation d'un mesh grossier � partir du lidar, color� � partir de plusieurs images, choix de l'image la plus en face :\verb"colormultiple.c"
\item z-buffer prenant en comptes les triangles, pas simplement les points du lidar
\item cr�ation d'un mesh le plus propre possible
\begin{itemize}
\item[\checkmark] r�gulariser le lidar (fait � l'aide du curvature microscope sur ipol)
\item (19/07) cr�er les faces triangulaires selon la hauteur des sommets et non toujours pareil
\item (19/07) regarder ``smoothing ... bilateral'' de Julie Digne sur ipol pour r�gulariser les normales des fa\c cades.
\end{itemize}
\item (19/07) recaler les images en utilisant les donn�es de gabriele sur menthe (menthe:/home/facciolo/iarpa/all_pairs/outdir_from_19_32/ncc_transform.txt)
\end{itemize}
 
 %%%%%%%%%%%%%%%%%
%%%%%%%REMARQUES%%%%%%%
 %%%%%%%%%%%%%%%%%
\section{Remarque}
\subsection{Cr�ation de mesh � partir de lidar}
Le lidar est bruit�, le bord des immeubles n'est pas lisse. Le mesh na�f obtenu en reliant les points du lidar n'est donc pas satisfaisant.
L'utilisation de ``The Image Curvature Microscope'' sur ipol permet d'obtenir un lidar plus lisse � partir duquel on obtient un mesh plus propre. Plut�t que de cr�er les triangles toujours dans le m�me ordre il faudrait aussi que cela d�pende de la hauteur des quatre points utilis�s. Cela donnerait des surfaces plus lisses pour les immeubles en diagonal sur le lidar.


 %%%%%%%%%%%%%%%%%
%%%%%%%AVANCEMENT%%%%%%%
 %%%%%%%%%%%%%%%%%
\section{R�sum� des activit�s de la journ�e}
\paragraph{19/07/2017} cr�ation de ce pdf, essai de \verb"colormultiple.c" sur le lidar r�gularis�




%------------------------------------------------------------------------------
%\section*{Acknowledgment}
%This work was partly founded by  Centre National d'Etudes Spatiales (CNES, MISS
%Project), European Research Council (advanced grant Twelve Labours), Office of
%Naval research (ONR grant N00014-14-1-0023), DGA St�r�o project,  ANR-DGA
%(project ANR-12-ASTR-0035), FUI (project Plein Phare) and Institut
%Universitaire de France.

%------------------------------------------------------------------------------
\bibliographystyle{siam}
\bibliography{biblio}

%\appendix




\end{document}

